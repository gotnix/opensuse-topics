\documentclass[adobefonts]{ctexart}
%\documentclass[winfonts]{ctexart}

\CTEXoptions[captiondelimiter={\quad}]


\usepackage{amsmath}            % AMS的数学宏包
\usepackage{amssymb}            % AMS的数学符号宏包
\usepackage{graphicx}           % 插入图片需要的宏包
\usepackage{float}              % 强大的浮动环境控制宏包
\usepackage{framed}             % `shaded'环境需要用到
\usepackage{enumitem}           % 增强列表功能
\usepackage{alltt}              % 在`alltt'环境中为等宽字体, 但可以使用LaTeX命令

% \usepackage{shortvrb}           % 简化\verb的写法
% \MakeShortVerb{\|}
\usepackage{listings}
\lstset{language=bash}
\lstset{extendedchars=false}
\lstset{breaklines}
\lstset{stepnumber=2}
\lstset{backgroundcolor=\color{lightgray}}


\usepackage{color}              % 可以定义各种颜色
\usepackage[x11names]{xcolor}   % 下面的RoyalBlue3颜色需要用到的宏包
% 自定义的几种颜色
\definecolor{shadecolor}{gray}{0.85}

% \definecolor{darkblue}{rgb}{52,101,164}
% \definecolor{darkgreen}{rgb}{78,154,6}

% % 设置背景颜色
% \definecolor{bisque}{rgb}{.996,.891,.755}
% \pagecolor{bisque}

\usepackage[pdfauthor={Dreamseeker},
  pdftitle={openSUSE 11.4 安装fcitx小企鹅输入法},
  colorlinks=true,
  urlcolor=blue,
  linkcolor=RoyalBlue3]{hyperref} % 为超链接设置颜色, 修改PDF文件信息

%\CTEXsetup[name={实验,},number={\chinese{section}}]{section}


\title{\textbf{openSUSE 11.4 安装fcitx小企鹅输入法}}
\author{kroodylove@gmail.com}
% \date{}

\usepackage[pagestyles]{titlesec} % 定制页眉页脚
% % 设置页眉页脚
% \newpagestyle{main}{%
%   \sethead[$\cdot$~\thepage~$\cdot$][][\thesection\quad%
%   \sectiontitle]{\thesection\quad\sectiontitle}{}{%
%   $\cdot$~\thepage~$\cdot$}
%   \setfoot{}{}{}\headrule}
% \pagestyle{main}
% \renewpagestyle{plain}{\sethead{}{}{}\setfoot{}{}{}}
\pagestyle{plain}

\usepackage[top=0.75in,bottom=0.5in,left=1in,right=1in]{geometry} % 设置页边距

\setlength{\belowcaptionskip}{1em} % 设置caption之后的距离

% For LaN
\newcommand{\LaN}{L{\scriptsize\hspace{-0.47em}\raisebox{0.23em}{A}}\hspace{-0.1em}N}
\begin{document}

\maketitle
\tableofcontents

\newpage
\section{我为什么使用fcitx}
其实自从去年我就在openSUSE11.3上开始使用fcitx(小企鹅)输入法了,原因很简单原来用默认的SCIM在跟朋友聊天的时候老得选字(用五笔的朋友飘过)特别麻烦,自带的词库有限啊。实在不如windows下的搜狗之类的来的爽,其实很早就关注fcitx了当时折腾了几下没好用就扔下了,过了很久才又重新拿出来配置起来用试试看吧,感觉还不错界面做的比原来好看很多而且速度快了很多,还支持换肤  统计打字速度等功能(甭管有用没用人家有就是亮点对吧)其他的要靠使用者去体会了。

我想写出来这个的原因也挺简单的,近期在openSUSE11.4发布后(什么 你不知道?!)

\href{http://www.opensuse.org}{www.opensuse.org}

社区里面很多朋友说自带的SCIM或者Ibus输入法或多或少有些问题。例如在firefox里面不好用,或者在Libreoffice里面有问题之类,虽然我机器上没发现上述现象(难道是RP好?),但还是写出来供大家参考一下吧。

文章放到了下面的地址,相信你能找到你所要的东东

\href{http://code.google.com/p/opensuse-topics/}{http://code.google.com/p/opensuse-topics/}

\section{安装配置fcitx}
\subsection{获得fcitx}
fcitx我原来装的是3.6版本的,现在已经更新到4.0
在上面的地址可以下载到源码和找到很多帮助,另外网上也一搜一堆的配置

\href{http://www.fcitx.org/main/}{http://www.fcitx.org/main/}

\href{http://code.google.com/p/fcitx/}{http://code.google.com/p/fcitx/}

\subsection{前期准备工作}

要注意的是如果你是英文默认安装完openSUSE11.4的话你还需要装几个输入法相关的包,如果和我一样懒就这么干在yast里面将language选成中文
这时候会装上SCIM和相关的包。如果已经是中文的就继续下一步:在命令行下切到root,删掉scim或者ibus输入法
\begin{verbatim}
zypper rm scim
\end{verbatim}
这样就可以了
语言看你喜好了 中文英文都可以,下面开始安装fcitx
\subsection{编译安装}
下面来正式安装拉
\subsubsection{第一步:(把冰箱门打开...)}
在终端命令行下到你源码所在目录:
\begin{verbatim}
tar zxf fcitx-4.0.1_all.tar.gz
cd fcitx-4.0.1/
./configure && make
\end{verbatim}
编译完成后sudo make install就可以了
\subsubsection{第二步:(把大象放冰箱...)} % (fold)
要修改系统几个配置文件,
vim ~/.bashrc加入下面三行
\begin{verbatim}
export XMODIFIERS="@im=fcitx"
export XIM=fcitx
export XIM_PROGRAM=fcitx
\end{verbatim}
切换到root
\begin{verbatim}
su
vim /etc/X11/xim
\end{verbatim}
改成这样
\begin{verbatim}
export XMODIFIERS="@im=fcitx"
\end{verbatim}
 保存退出
上面XIM在系统注册的名字为fcitx。应用程序启动的时候会根据该变量查找相应的XIM服务器。
下一个文件是
\begin{verbatim}
vim /etc/sysconfig/language 
INPUT_METHOD="fcitx"
\end{verbatim}
然后进入到目录/etc/X11/xim.d
添加一个名字为fcitx的文件,内容如下
\begin{verbatim}
OLD_PATH=$PATH
PATH=/usr/local/bin:/usr/bin:/opt/kde3/bin:$PATH

if ! type -p fcitx > /dev/null 2>&1 ; then
    echo "fcitx is not available."
    return 1
fi

export XMODIFIERS=@im=fcitx
export GTK_IM_MODULE=fcitx
export QT_IM_SWITCHER=imsw-multi
export QT_IM_MODULE=fcitx
case "$WINDOWMANAGER" in
    /opt/kde3/bin/startkde)
        if ! type -p skim > /dev/null 2>&1 \
          || grep -i -q "^[[:space:]]*Autostart.*=.*false" $HOME/.kde/share/config/skimrc
        then
            fcitx -d
        else
        # skim will be  used. But we don't start it here,
        # we rely on the KDE3 autostart
        # mechanism used in the skim package instead.
        # skim -d
        :
        fi
    ;;
    *)
        fcitx -d
    ;;
esac

PATH=$OLD_PATH

# success:
return 0

\end{verbatim}
\subsubsection{第三步:(把冰箱门带上...)}
好了,还有最后一个地方,我们要让每次进桌面让输入法启动起来才可以
鼠标点击Computer,选Control Center---Startup Applications(中文界面的话自己翻译着看吧就是开始菜单 控制中心 里面的会话拉)

点添加,名字随便起 就叫fcitx吧 运行命令就是
\begin{verbatim}
/usr/local/bin/fcitx(编译时候./configure 是默认路径就是这个)
\end{verbatim}
保存退出就可以了
\input{../readme}
\end{document}
